\documentclass[12pt,a4paper]{report}
\usepackage[utf8]{inputenc} % this is needed for umlauts
\usepackage[ngerman]{babel} % this is needed for umlauts
\usepackage[T1]{fontenc}    % this is needed for correct output of umlauts in pdf

\usepackage{ucs}
\usepackage{amsmath}
\usepackage{amsfonts}
\usepackage{amssymb}
\usepackage{graphicx}

\usepackage{hyperref}
\usepackage{csquotes}
\usepackage{appendix}
\usepackage{pdfpages}
\usepackage{float}

\usepackage[weather]{ifsym}

\begin{document}

\newcommand \GruppeA {Lerngruppe Henri Libre}

	\tableofcontents
	\newpage
	
	\chapter{Vorwort}
	Dieses Dokument findest du auf github.com unter: \href{https://github.com/henri-libre/analysis1}{https://github.com/henri-libre/analysis1}. Du darfst das Dokument nutzen, erweitern und verbreiten. Maintainer des Dokumentes erreichst du entweder dort oder per E-Mail an \href{analysis1istgeil@nanooq.org}{analysis1istgeil@nanooq.org}. Für die Korrektheit des Dokumentes ist entweder keiner oder du verantwortlich. Die URL der Veranstaltung an sich lautet: \href{https://analysis3.wordpress.com/analysis-i-ws-1516/uebungen-zu-analysis-i-wise-1516/}{https://analysis3.wordpress.com/analysis-i-ws-1516/uebungen-zu-analysis-i-wise-1516/}
	
	\setcounter{chapter}{1}
\chapter*{Übungsblatt 1}
	
	Das entsprechende Übungsblatt befindet sich im Anhang \ref{uebungsblatt1} \footnote{\href{https://analysis3.files.wordpress.com/2015/08/blatt1.pdf}{https://analysis3.files.wordpress.com/2015/08/blatt1.pdf}}.

\newpage		
\section{Übungsblatt1, Aufgabe 1: Vollständige Induktion, \GruppeA}
	
	\begin{enumerate}
	\item Z.~z.: $5^n - 1 | 4 $ 
		\begin{itemize}
			\item Induktionsanker:
			\begin{flalign*}
				n_1 = 1: ~~& 5^1 - 1 = 4 ~~ \checkmark &\\
				n_2 = 2: ~~& 5^2 - 1 = 24~~ \checkmark &\\
				n_3 = 3: ~~& 5^3 - 1 = 124~~ \checkmark
			\end{flalign*}
			\item Induktionsvoraussetzung:
			\begin{flalign*}
				(5^n -1) | 4 &&
			\end{flalign*}
			\item Induktionsschritt:
			\begin{flalign*}
				n \mapsto n+1: ~~ & 5^{n+1}-1 & \\ 
				= & (5 \cdot 5^n)-1 & \\
				= & (4 \cdot 5^n + 5^n) -1 & \\
				= & (4 \cdot 5^n) + (5^n -1)
			\end{flalign*}
			Erster Term ist durch 4 teilbar. Zweiter Term ist gleich unserer Induktionsvoraussetzung.  
		\end{itemize}
	\newpage
	\item Z.~z.: $ 3^{2^n}-1|2^{n+2} $ 
		\begin{itemize}
			\item Induktionsanker:
			\begin{flalign*}
				n_1 = 1: ~~& 3^{2^1} - 1 | 2^{1+2} \Leftrightarrow 8 | 8 ~~ \checkmark & \\
				n_2 = 2: ~~& 3^{2^2} - 1 | 2^{2+2} \Leftrightarrow 80 | 16 ~~ \checkmark & \\
				n_3 = 3: ~~& 3^{2^3} - 1 | 2^{3+2} \Leftrightarrow 6560 | 32 ~~ \checkmark
			\end{flalign*}
			\item Induktionsvoraussetzung:
			\begin{flalign*}
				3^{2^n} - 1 | 2^{n+2} &&
			\end{flalign*}
			\item Induktionsschritt:
			\begin{flalign*}
				n \mapsto n+1: ~~ & 3^{2^{n+1}} -1 & \\
				 \overset{\text{aus Klammer ziehen}}{=} & (3^{2^n})^2 -1 &\\ 
				 \overset{\text{3. Binomische Formel}}{=} & (3^{2^n}-1)(3^{2^n}+1) 
			\end{flalign*}

			Erster Term ist die Induktionsvoraussetzung. Der zweite Term ist im Detail unwichtig, wegen der Multiplikation.
		\end{itemize}
	\item Z.~z.: Für Menge $M$ mit $|M| = n \Rightarrow |P(M)|=2^n$
		\begin{itemize}
			\item Induktionsvoraussetzung:
			$ n=1: $ Sei $M=\{n\}$. Dann ist $P(M)=\{\varnothing,\{a\}\} \rightarrow |P(M)|=2^1$.
			\item Induktionsschritt:\\
			Dann sei $M^*=\{a_1,\dots, a_n\}$.\\
			Dann gilt laut Induktionsvoraussetzung: $|P(M^*)=2^n|$.
			\\Nun gilt\footnote{$P(M) \setminus P(M^*)$ muss disjunkt sein, weil: Wenn A, B disjunkt, dann $ A \cap B = \varnothing$.} $P(M) \setminus P(M^*) = \{T \cup \{a_{n+1}\} | T \in P(M^*)\} \\
			\Rightarrow |P(M)|=2|P(M^*)|=2\cdot 2^n= 2^{n+1}$
			\item Erklärung:\\
			Es sei $ P(m^*) = \{\varnothing\}, \{a_1\}, \{a_2\}, \{a_1,a_2\}$. Wenn nun $ \{a_3\} $ hinzugefügt wird, ist $ P(m) = P(m^*) \cup \{a_{3}\} = \{\varnothing\}, \{a_1\}, \{a_2\}, \{a_1,a_2\},\\
			\{a_3\}, \{a_1,a_3\}, \{a_2,a_3\}, \{a_1,a_2,a_3\} $.
		\end{itemize}
	\end{enumerate}

\newpage
\section{Übungsblatt 1, Aufgabe 2: Indirekter Beweis, \GruppeA}
		
	Zeige zunächst: $ (1+k) \leq 2^k $ ~~ $ \forall k \in N$
	\begin{itemize}
		\item Induktionsanker:
		\begin{flalign*} 
		k_1 = 1: ~~ & 1 + 1 \leq 2^1 \Leftrightarrow 2 \leq 2 ~~ \checkmark & \\
		k_2 = 2: ~~ & 1 + 2 \leq 2^2 \Leftrightarrow 3 \leq 4 ~~ \checkmark & \\
		k_3 = 3: ~~ & 1 + 3 \leq 2^3 \Leftrightarrow 4 \leq 8 ~~ \checkmark
		\end{flalign*}
		\item Induktionsvoraussetzung:
		\begin{flalign*}
			(1+k) \leq 2^k ~~ \forall k \in N &&
		\end{flalign*}
		\item Induktionsschritt:
			\begin{flalign*}
			k \mapsto k+1: ~~  1+(k+1) &\leq 2^{k +1} & \\
			\Leftrightarrow (2+k) &\leq 2 \cdot 2^k & \\
			\Leftrightarrow \frac{2+k}{2} &\leq (1+k) \leq 2^k
			\end{flalign*}
			Der erste Term ist immer kleiner als die linke Seite der Induktionsvoraussetzung.
		\end{itemize}

	Exkurs \enquote{Indirekter Beweis}: $(A \Rightarrow B) \Leftrightarrow (\neg B \Rightarrow \neg A)$:

	\begin{table}[H]
		\center
		\begin{tabular}{c | c | c | c | c }
			\# & A & B & A $\Rightarrow B $ & $ \neg B \Rightarrow \neg A $ \\
			\hline
			1 & 0 & 0 & 1 & 1 \\
			2 & 0 & 1 & 1 & 1 \\
			3 & 1 & 0 & 0 & 0 \\
			4 & 1 & 1 & 1 & 1 \\
		\end{tabular}
		\caption{In der dritten Zeile steht der Indirekter Beweis}
	\end{table}
	
	Indirekter Beweis: Z.~z.: Für $ a_1, \ldots, a_n \text{ mit } n \in N$ gilt:\\
	\begin{align*}
	\prod^{n}_{i=1}(1+a_i) > 2^n\text{, so folgt} \sum^{n}_{i=1}a_i > n &&
	\end{align*}
	Eigentlich sieht das so schöner aus:
	\begin{align*}
	n < \sum^{n}_{i=1}a_i < 2^n < \prod^{n}_{i=1}(1+a_i)&&
	\end{align*}	
	
	\begin{itemize}
		\item \enquote{Anker}. Einfach mal durchrechnen:
		\begin{flalign*}
			n_1 = 1: ~~ & 1 < 1 < 2^1 < (1 + 1) ~~ \text{\Lightning} & \\
			n_2 = 2: ~~ & 2 < 1+2 < 2^2 < (1+1)(1+2) ~~ \checkmark & \\
			n_3 = 3: ~~ & 3 < 1+2+3 < 2^3 < (1+1)(1+2)(1+3) ~~ \checkmark &
		\end{flalign*}
		\item \enquote{Voraussetzung}, zu zeigen:
			\begin{flalign*}
			n < \sum^{n}_{i=1}a_i < 2^n < \prod^{n}_{i=1}(1+a_i) &&	
			\end{flalign*}
		\item Indirekter Beweis:
			\begin{flalign*}
			& \prod^{n}_{i=1}(1+a_i) = (1+a_1) \cdot \ldots \cdot (1+a_n) & \\
			& \text{jeder Term kleiner als} & \\
			& \leq 2^{a_1} \cdot \ldots \cdot 2^{a_n} = 2^{\sum_{i=1}^{n}a_i} &\\
			& \overset{\text{vorhin bewiesen}}{\leq} 2^n & \\ 
			\end{flalign*}
	\end{itemize}
\newpage
\section{Übungsblatt 1, Aufgabe 3: Vollständige Induktion, \GruppeA}

	\begin{enumerate}
		\item Z.~z. dass das Brett mit \enquote{L}-förmigen Kartonstücken überdecken.
		\begin{itemize}
			\item Induktionsanker:
				\begin{table}[H]
					\centering
					\begin{tabular}{|c|c|c|c|}
						\hline
						~~&~~&~~&~~\\
						\hline
						&&&\\
						\hline
						&&&\\
						\hline
						&&&\\
						\hline
					\end{tabular}
					\caption{Schachbrett mit der Seitenläge $2^n$}
				\end{table}
			\item Induktionsvoraussetzung:\\
			$n=1: 2 \times 2 \rightsquigarrow$ \enquote{L}-förmiges Kartonstückchen $\Rightarrow$ Vollständige Überdeckung.
			
			\item Induktionsschritt:\\
			$n\mapsto n+1$: Wir nehmen von einem Schachbrett mit Seitenlänge $2^{n+1}$ ein Feld weg und teilen es in 4 Schachbretter mit der Seitenlänge $2^n$ auf:
			\begin{table}[H]
				\centering
				\begin{tabular}{|c|c||c|c|}
					\hline
					2 & 2 & 1 & X \\
					\hline
					2 & 2 & 1 & 1 \\
					\hline
					\hline
					3 & 3 & 4 & 4 \\
					\hline
					3 & 3 & 4 & 4 \\
					\hline
				\end{tabular}
				\caption{Schachbrett mit der Seitenläge $2^{n+1} $, das weggenommene Feld sei X}
			\end{table}
			Sei das weggenommene Feld ohne Beschränkung der Allgemeinheit oben rechts aus 1. Dann können wir laut Induktionsvoraussetzung das Feld 1 vollständig überdecken. Genauso können wir dies mit den Feldern 2, 3 und 4 bis auf ein Feld überdecken. Diese drei Felder werden so angeordnet:
			
			\begin{table}[H]
				\centering
				\begin{tabular}{|c|c||c|c|}
					\hline
					2 & 2 & 1 & X \\
					\hline
					2 & L & 1 & 1 \\
					\hline
					\hline
					3 & L & L & 4 \\
					\hline
					3 & 3 & 4 & 4 \\
					\hline
				\end{tabular}
				\caption{}
			\end{table}
			
			Dann können wir diese drei Felder auch überdecken. Dann ist das gesamte Brett mit Ausnahme des weggenommenen Feldes vollständig überdeckt. $ \square $
		\end{itemize}
	\item Aus dem ersten Aufgabenteil folgt nicht, dass der Beweis für diesen Aufgabenteil. Viel mehr ist der erste Teil ein Spezialfall von diesem Aufgabenteil.\\
	Z.~z.: $ 3|2^{2n}-1$
	\begin{itemize}
		\item Induktionsanker:
		\begin{flalign*}
			n_1 = 1:~~ & 2^{2\cdot 1} - 1 = 3  ~~\checkmark &\\
			n_2 = 2:~~ & 2^{2\cdot 2} - 1 = 15 ~~\checkmark &\\
			n_3 = 3:~~ & 2^{2\cdot 3} - 1 = 63 ~~\checkmark &
		\end{flalign*}
		\item Induktionsvoraussetzung:
		\begin{flalign*}
			3|2^{2n}-1 &&
		\end{flalign*}
		\item Induktionsschritt:
		\begin{flalign*}
			n \mapsto n+1: ~~ & 2^{2(n+1)} - 1 & \\
			& = 2^{2n+2} - 1 & \\
			& = (2^{2n} - 1 + 1) \cdot 4 - 1 & \\
			& = (2^{2n} - 1) \cdot 4 + 1 \cdot 4 - 1 & \\
			& = 4(2^{2n} - 1) + 3
		\end{flalign*}
		Der letzte Term ist durch 3 teilbar.  Die Multiplikation am Anfang ist ein Vierfaches der Induktionsvoraussetzung. Die Induktionsvoraussetzung selber ist durch 3 teilbar. Also ist auch die Multiplikation durch 3 teilbar. Damit ist die ganze Formel durch 3 teilbar.
	\end{itemize}
	\end{enumerate}

\newpage
\section{Übungsblatt 1, Aufgabe 4: Vollständige Induktion, \GruppeA}
Der Fehler in dem Induktionsverwis liegt im Folgenden: Innerhalb des Induktionsschrittes nimmt die Induktionan, dass die Induktionsannahme gleich der Behauptung, also wahr, ist. Dieser Schritt ist jedoch fehlerhaft, da nie bewiesen wurde, on jede Aussage wahr ist. Da man von einer falschen Aussage auf sowohl wahre als auch falsche Aussage schließen kann, fehlt daher ein Berweis, dass die Induktionsanname wahr ist. Ein Gegenbeweis wäre z.~B. für $n=2$.

\newpage
\section{Musterlösung}

Stand 2015-10-29: Noch nicht bekannt gegeben

\appendices
	\chapter{}
	\section{Übungsblatt 1}\label{uebungsblatt1}
	\includepdf[pages=-]{blatt1}
\end{document}